% Discuss the strengths and weaknesses of your approach, based on the results. Point out the implications of your novel idea on the application concerned.

\section{Discussion} \label{section:discussion}

We observe that an accurately tuned U-Net performs similar to the more complex \acrshort{gcdcnn} and other variants with slight architecture alterations, suggesting that the exact model architecture plays a minor role. The largest contribution to prediction accuracy was reached by using additional diverse training images. Averaging multiple predictions in an ensemble significantly improves the predictions by removing artefacts and by combining the strengths of several models. Despite not improving Kaggle score significantly, our proposed post-processing method: retraining on binary images with a large receptive field was able to fill small gaps between roads. Since we are evaluated on the patch-level accuracy, these slight improvements did not reflect in the patched predictions. A general drawback of post-processing techniques is their required extra work to train another network.

%We observed that an accurately tuned U-Net performs similarly good, even better than the more complex GC-DCNN and other variants with slight architecture alternations. The observed differences in accuracy are very small, suggesting that the exact model architecture plays only a minor role. The largest contribution to prediction accuracy was reached by using additional training images which are as diverse as possible. Averaging multiple predictions in an ensemble significantly improves the predictions by removing artefacts and by combining the strengths of several models.

%Even though not all of our proposed post-processing techniques were able to improve prediction accuracy, retraining on binary images with a large receptive field was able to fill small gaps between roads and therefore enhance the visual quality of the predictions. Since we are evaluated on the patch-level accuracy, these slight improvements did not reflect on the patch predictions. A general drawback of post-processing techniques is their required extra work to train another network.